% Options for packages loaded elsewhere
\PassOptionsToPackage{unicode}{hyperref}
\PassOptionsToPackage{hyphens}{url}
%
\documentclass[
  ignorenonframetext,
  aspectratio=169]{beamer}
\usepackage{pgfpages}
\setbeamertemplate{caption}[numbered]
\setbeamertemplate{caption label separator}{: }
\setbeamercolor{caption name}{fg=normal text.fg}
\beamertemplatenavigationsymbolsempty
% Prevent slide breaks in the middle of a paragraph
\widowpenalties 1 10000
\raggedbottom
\setbeamertemplate{part page}{
  \centering
  \begin{beamercolorbox}[sep=16pt,center]{part title}
    \usebeamerfont{part title}\insertpart\par
  \end{beamercolorbox}
}
\setbeamertemplate{section page}{
  \centering
  \begin{beamercolorbox}[sep=12pt,center]{part title}
    \usebeamerfont{section title}\insertsection\par
  \end{beamercolorbox}
}
\setbeamertemplate{subsection page}{
  \centering
  \begin{beamercolorbox}[sep=8pt,center]{part title}
    \usebeamerfont{subsection title}\insertsubsection\par
  \end{beamercolorbox}
}
\AtBeginPart{
  \frame{\partpage}
}
\AtBeginSection{
  \ifbibliography
  \else
    \frame{\sectionpage}
  \fi
}
\AtBeginSubsection{
  \frame{\subsectionpage}
}
\usepackage{lmodern}
\usepackage{amssymb,amsmath}
\usepackage{ifxetex,ifluatex}
\ifnum 0\ifxetex 1\fi\ifluatex 1\fi=0 % if pdftex
  \usepackage[T1]{fontenc}
  \usepackage[utf8]{inputenc}
  \usepackage{textcomp} % provide euro and other symbols
\else % if luatex or xetex
  \usepackage{unicode-math}
  \defaultfontfeatures{Scale=MatchLowercase}
  \defaultfontfeatures[\rmfamily]{Ligatures=TeX,Scale=1}
\fi
\usetheme[]{Frankfurt}
\usecolortheme{beaver}
% Use upquote if available, for straight quotes in verbatim environments
\IfFileExists{upquote.sty}{\usepackage{upquote}}{}
\IfFileExists{microtype.sty}{% use microtype if available
  \usepackage[]{microtype}
  \UseMicrotypeSet[protrusion]{basicmath} % disable protrusion for tt fonts
}{}
\makeatletter
\@ifundefined{KOMAClassName}{% if non-KOMA class
  \IfFileExists{parskip.sty}{%
    \usepackage{parskip}
  }{% else
    \setlength{\parindent}{0pt}
    \setlength{\parskip}{6pt plus 2pt minus 1pt}}
}{% if KOMA class
  \KOMAoptions{parskip=half}}
\makeatother
\usepackage{xcolor}
\IfFileExists{xurl.sty}{\usepackage{xurl}}{} % add URL line breaks if available
\IfFileExists{bookmark.sty}{\usepackage{bookmark}}{\usepackage{hyperref}}
\hypersetup{
  pdftitle={Biodiversity Database and Biodiversity Indexing},
  pdfauthor={Deependra Dhakal},
  hidelinks,
  pdfcreator={LaTeX via pandoc}}
\urlstyle{same} % disable monospaced font for URLs
\newif\ifbibliography
\setlength{\emergencystretch}{3em} % prevent overfull lines
\providecommand{\tightlist}{%
  \setlength{\itemsep}{0pt}\setlength{\parskip}{0pt}}
\setcounter{secnumdepth}{-\maxdimen} % remove section numbering
\usepackage{booktabs}
\usepackage{longtable}
\usepackage{array}
\usepackage{multirow}
\usepackage{wrapfig}
\usepackage{float}
\usepackage{colortbl}
\usepackage{pdflscape}
\usepackage{tabu}
\usepackage{threeparttable}
\usepackage{threeparttablex}
\usepackage[normalem]{ulem}
\usepackage{makecell}
\usepackage{xcolor}
\usepackage{tikz} % required for image opacity change
\usepackage[absolute,overlay]{textpos} % for text formatting

% this font option is amenable for beamer
\setbeamerfont{caption}{size=\tiny}

\title{Biodiversity Database and Biodiversity Indexing}
\author{Deependra Dhakal}
\date{}
\institute{College of Natural Resource Management, Tikapur,
Kailali \and Agriculture and Forestry University}

\begin{document}
\frame{\titlepage}

\begin{frame}[allowframebreaks]
  \tableofcontents[hideallsubsections]
\end{frame}
\hypertarget{indices}{%
\section{Indices}\label{indices}}

\begin{frame}{Meaning}
\protect\hypertarget{meaning}{}
\footnotesize

A diversity \textbf{index} is a quantitative measure that reflects
\textbf{how many different types} (such as species) there are in a
dataset (a community), and that can simultaneously take into account the
phylogenetic relations among the individuals distributed among those
types, such as richness, divergence or evenness. These indices are
statistical representations of biodiversity in different aspects
(richness, evenness, and dominance).

When diversity indices are used in ecology, the types of interest are
usually species, but they can also be other categories, such as genera,
families, functional types, or haplotypes. The entities of interest are
usually individual plants or animals, and the \textbf{measure of
abundance} can be, for example, number of individuals, biomass or
coverage.

The most commonly used diversity indices are simple transformations of
the \textbf{effective number of types} (also known as `\textbf{true
diversity}').
\end{frame}

\begin{frame}{Effective number of species}
\protect\hypertarget{effective-number-of-species}{}
\footnotesize

\textbf{True diversity}, or the \textbf{effective number of types},
refers to the number of equally abundant types needed for the average
proportional abundance of the types to equal that observed in the
dataset of interest (where all types may not be equally abundant). The
true diversity in a dataset is calculated by first taking the weighted
generalized mean \(M_{q-1}\) of the proportional abundances of the types
in the dataset, and then taking the reciprocal of this. The equation is:

\[
{}^{q}\!D={1 \over M_{q-1}}=
{1 \over {\sqrt[{q-1}]{\sum_{i=1}^{R}p_{i}p_{i}^{q-1}}}}=\left({\sum_{i=1}^{R}p_{i}^{q}}\right)^{1/(1-q)}
\]

The denominator \(M_{q-1}\) equals the average proportional abundance of
the types in the dataset as calculated with the weighted generalized
mean with exponent \(q-1\). In the equation, R is richness (the total
number of types in the dataset), and the proportional abundance of the
\(i\)th type is \(p_i\). The proportional abundances themselves are used
as the nominal weights. The numbers \({}^{q}D\) are called Hill numbers
of order \(q\) or effective number of species.
\end{frame}

\begin{frame}{Shannon entropy}
\protect\hypertarget{shannon-entropy}{}
When q = 1, the above equation is undefined. However, the mathematical
limit as q approaches 1 is well defined and the corresponding diversity
is calculated with the following equation:

\[
{}^{1}\!D={1 \over {\prod _{i=1}^{R}p_{i}^{p_{i}}}}=\exp \left(-\sum _{i=1}^{R}p_{i}\ln(p_{i})\right)
\]

which is the exponential of the \textbf{Shannon entropy} calculated with
natural logarithms.
\end{frame}

\begin{frame}{General Equation}
\protect\hypertarget{general-equation}{}
\[
{}^{q}\!D=\left({\sum _{i=1}^{R}p_{i}^{q}}\right)^{1/(1-q)}
\]

The term inside the parentheses is called the basic sum. Some popular
diversity indices correspond to the basic sum as calculated with
different values of q.
\end{frame}

\begin{frame}{Sensitivity of diversity value to rare versus abundant
species}
\protect\hypertarget{sensitivity-of-diversity-value-to-rare-versus-abundant-species}{}
\footnotesize

It defines the sensitivity of the true diversity to rare vs.~abundant
species by modifying how the weighted mean of the species' proportional
abundances is calculated. With some values of the parameter \(q\), the
value of the generalized mean \(M_{q-1}\) assumes familiar kinds of
weighted means as special cases. In particular,

\begin{itemize}
\tightlist
\item
  \(q = 0\) corresponds to the weighted harmonic mean,
\item
  \(q = 1\) to the weighted geometric mean, and
\item
  \(q = 2\) to the weighted arithmetic mean.
\item
  As \(q\) approaches infinity, the weighted generalized mean with
  exponent \(q-1\) approaches the maximum \(p_i\) value, which is the
  proportional abundance of the most abundant species in the dataset.
\end{itemize}

Generally, increasing the value of \(q\) increases the effective weight
given to the most abundant species. This leads to obtaining a larger
\(M_{q-1}\) value and a smaller true diversity (\({}^q D\)) value with
increasing \(q\).
\end{frame}

\begin{frame}{Richness}
\protect\hypertarget{richness}{}
\begin{itemize}
\tightlist
\item
  Richness \(R\) simply quantifies how many different types the dataset
  of interest contains. For example, species richness (usually noted
  \(S\)) of a dataset is the number of species in the corresponding
  species list.
\item
  Richness is a simple measure, so it has been a popular diversity index
  in ecology, where abundance data are often not available for the
  datasets of interest.
\item
  Because richness does not take the abundance of the types into
  account, it is not the same thing as diversity, which does take
  abundance into account.
\item
  If true diversity is calculated with \(q = 0\), the effective number
  of types (\({}^0 D\)) equals the actual number of types, which is
  identical to Richness (\(R\)).
\end{itemize}
\end{frame}

\begin{frame}{Shannon Index}
\protect\hypertarget{shannon-index}{}
\begin{itemize}
\tightlist
\item
  The Shannon index = Shannon's diversity index = Shannon--Wiener index
  \(\neq\) Shannon--Weaver index (erroneous).
\item
  The measure was originally proposed by Claude Shannon in 1948 to
  quantify the entropy.
\end{itemize}

It is most often calculated as follows:

\[
H'=-\sum _{i=1}^{R}p_{i}\ln p_{i}
\] \(p_i\) is often the proportion of individuals belonging to the
\(i\)th species in the dataset of interest. Then the Shannon entropy
quantifies the uncertainty in predicting the species identity of an
individual that is taken at random from the dataset.
\end{frame}

\begin{frame}{}
\protect\hypertarget{section}{}
The Shannon index (\(H'\)) is related to the weighted geometric mean of
the proportional abundances of the types. Specifically, it equals the
logarithm of true diversity as calculated with \(q = 1\):

\[
H'=-\sum _{i=1}^{R}p_{i}\ln p_{i}=-\sum _{i=1}^{R}\ln p_{i}^{p_{i}}
\] This can also be written

\[
H'=-(\ln p_{1}^{p_{1}}+\ln p_{2}^{p_{2}}+\ln p_{3}^{p_{3}}+\cdots +\ln p_{R}^{p_{R}})
\]

which equals,
\end{frame}

\begin{frame}{}
\protect\hypertarget{section-1}{}
\footnotesize

\[
H'=-\ln p_{1}^{p_{1}}p_{2}^{p_{2}}p_{3}^{p_{3}}\cdots p_{R}^{p_{R}}=\ln \left({1 \over p_{1}^{p_{1}}p_{2}^{p_{2}}p_{3}^{p_{3}}\cdots p_{R}^{p_{R}}}\right)=\ln \left({1 \over {\prod _{i=1}^{R}p_{i}^{p_{i}}}}\right)
\]

Since the sum of the pi values equals unity by definition, the
denominator equals the weighted geometric mean of the \(p_i\) values,
with the \(p_i\) values themselves being used as the weights (exponents
in the equation). The term within the parentheses hence equals true
diversity \({}^1 D\), and \(H'\) equals \(ln({}^1 D)\).

When all types in the dataset of interest are equally common, all
\(p_i\) values equal \(\frac{1}{R}\), and the Shannon index hence takes
the value \(ln(R)\). The more unequal the abundances of the types, the
larger the weighted geometric mean of the \(p_i\) values, and the
smaller the corresponding Shannon entropy. If practically all abundance
is concentrated to one type, and the other types are very rare (even if
there are many of them), Shannon entropy approaches zero. When there is
only one type in the dataset, Shannon entropy exactly equals zero (there
is no uncertainty in predicting the type of the next randomly chosen
entity).
\end{frame}

\hypertarget{numerical-problems}{%
\section{Numerical problems}\label{numerical-problems}}

\begin{frame}{Problem}
\protect\hypertarget{problem}{}
\begin{enumerate}
\tightlist
\item
  Calculate Shannon Weaner Index, Simpson Index and Evenness from
  following two areas and interpret your result. Also discuss species
  richness and evenness from following information.
\end{enumerate}

\begin{table}

\caption{\label{tab:insect-area-comparison}Data of insect counts from two areas.}
\centering
\fontsize{8}{10}\selectfont
\begin{tabular}[t]{llllll}
\toprule
Wasp & Butterfly & Grasshopper & Bettle & Bee & Hoverfly\\
\midrule
112 & 88 & 143 & 112 & 100 & 145\\
424 & 76 & 54 & 60 & 40 & 46\\
\bottomrule
\end{tabular}
\end{table}
\end{frame}

\begin{frame}{Solution}
\protect\hypertarget{solution}{}
\begin{columns}[T,onlytextwidth]
  \column{0.5\textwidth}

Here, 

We have Shannon-Weiner Index,

$$ 
H = \sum_{i = 1}^S{p_i.\ln p_i}
$$

Where, $S$ is the species richness (number of distinct species types), $p_i$ is the proportional abundance, and $\ln p_i$ the natural log of the proportional abundance.

  \column{0.5\textwidth}

\begin{table}

\caption{\label{tab:insect-area-index-calculation}Computed diversity indices for insect count data.}
\centering
\fontsize{8}{10}\selectfont
\begin{tabular}[t]{lrr}
\toprule
Index & Area 1 & Area 2\\
\midrule
shannon\_weiner & 1.78 & 1.30\\
simpson & -0.17 & -0.40\\
difference\_simpson & 1.17 & 1.40\\
richness & 6.00 & 6.00\\
evenness & 0.99 & 0.72\\
\bottomrule
\end{tabular}
\end{table}

\end{columns}
\end{frame}

\end{document}
